%% LyX 1.6.9 created this file.  For more info, see http://www.lyx.org/.
%% Do not edit unless you really know what you are doing.
\documentclass[master.tex]{subfiles}

\makeatletter
%%%%%%%%%%%%%%%%%%%%%%%%%%%%%% Textclass specific LaTeX commands.
\newcommand{\lyxaddress}[1]{
\par {\raggedright #1
\vspace{1.4em}
\noindent\par}
}
\newcommand{\lyxrightaddress}[1]{
\par {\raggedleft \begin{tabular}{l}\ignorespaces
#1
\end{tabular}
\vspace{1.4em}
\par}
}

\makeatother

\begin{document}

\title{Hidrodinamikai fókuszálás multilayer mikrofluidikai chip segítségével}

\author{Laki András József}

\maketitle

\lyxaddress{Konzulens: Dr. Iván Kristóf, Dr. Danilo Demarchi}

\lyxrightaddress{}

Korunkban egyre nagyobb igény mutatkozik a különböző szakterületek összekapcsolására. Az egyes diszciplínák elérték fejlődésük határait. Az új és áttörő fejlődéshez nélkülözhetetlen az egyes területeket egymásba integrálni. Az informatika robbanásszerű fejlődésével majdnem minden kutatási irány intenzív fejlődésbe kezdett. Legfőképpen az orvostudomány által életre hívott bioinformatikai ág indult virágzásnak. Célkitűzésünk ennek megfelelően több diszcíplinát áthidaló, Micro-Total-Analysis-Systems (uTAS) elvű, Lab-on-a-chip (LOC) mikrochip szerepelt. 

A Szent István Egyetem Állatorvosi Kar Parazitológiai és Állattani Tanszékén felmerült bioorientált mikrofluidikai feladathoz sikeresen terveztünk és összeállítottunk egy vizsgáló rendszert, mely segítségével mikronos nagyságú részecskéket vizsgálhatunk hidrodinamika segítségével. A legyártott, mikrofluidikai fókuszáló, multilayer mikrochip által tetszőleges nagyságú fluidum csatornát alakíthatunk ki pár mikronos nagyságban. Az üveg-szilícium-üveg mikrochiphez optikai rendszer is társítható. Mival a mikrocsatornán millilitereket akarunk minél rövidebb idő alatt átáramoltatni ezért a fluidum sebessége és nyomása nagyban megnő. A nagysebességű folyadék vizsgálatához nagy frame/s-al bíró kamerarendszer szükséges. 

További kísérleteink során rendszerünkhöz CNN univerzális gépet szeretnénk társítani, amely több ezer frame/s-os teljesítménnyel bír. Illetve elektródákat helyeztünk el a mikrocsatorna falaiba. Ezek segítségével érzékelhetjük a csatornában áthaladó részecskék által megváltoztatott elektromágneses teret, így egy újabb lehetőség nyílik a részecskék analizálására és összeszámolására. 

\end{document}